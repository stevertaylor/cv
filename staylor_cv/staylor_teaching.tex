%%%%%%%%%%%%%%%%%%%%%%%%%%%%%%%%%%%%%%%%%
% "ModernCV" CV and Cover Letter
% LaTeX Template
% Version 1.1 (9/12/12)
%
% This template has been downloaded from:
% http://www.LaTeXTemplates.com
%
% Original author:
% Xavier Danaux (xdanaux@gmail.com)
%
% License:
% CC BY-NC-SA 3.0 (http://creativecommons.org/licenses/by-nc-sa/3.0/)
%
% Important note:
% This template requires the moderncv.cls and .sty files to be in the same
% directory as this .tex file. These files provide the resume style and themes
% used for structuring the document.
%
%%%%%%%%%%%%%%%%%%%%%%%%%%%%%%%%%%%%%%%%%

%----------------------------------------------------------------------------------------
%   PACKAGES AND OTHER DOCUMENT CONFIGURATIONS
%----------------------------------------------------------------------------------------

\documentclass[11pt,letterpaper,sans]{moderncv} % Font sizes: 10, 11, or 12; paper sizes: a4paper, letterpaper, a5paper, legalpaper, executivepaper or landscape; font families: sans or roman

\moderncvstyle{banking} % CV theme - options include: 'casual' (default), 'classic', 'oldstyle' and 'banking'
\moderncvcolor{blue} % CV color - options include: 'blue' (default), 'orange', 'green', 'red', 'purple', 'grey' and 'black'
%\usepackage[scale=0.8]{geometry} % Reduce document margins
\usepackage[margin=1in]{geometry}
%\setlength{\hintscolumnwidth}{3cm} % Uncomment to change the width of the dates column
%\setlength{\makecvtitlenamewidth}{10cm} % For the 'classic' style, uncomment to adjust the width of the space allocated to your name

\def\prd{{Phys. Rev.} D}
\def\PRL{{Phys.Rev.} Lett}
\def\apjl{{Astrophys. J.} Lett}
\def\apj{{Astrophys. J.}}
\def\CQG{{Class. Quantum Grav.}}
\def\aaps{{A\&AS}}
\def\pasj{{PASJ}}
\def\mnras{{MNRAS}} 
\def\aapr{{A\&ARv}}
\def\aap{{A\&A}}
\def\na{{New Astronomy}}
\def\ptp{{Progress of Theoretical Physics}}
\def\apjs{{ApJS}}
\def\araa{{ARA\&A}}
\def\ssr{{Space Sci. Rev.}} 

\usepackage{etoolbox}% http://ctan.org/pkg/etoolbox
\makeatletter
\newcommand*{\emailA}[1]{\def\@emailA{#1}}
\newcommand*{\emailB}[1]{\def\@emailB{#1}}
\patchcmd{\maketitle}% <cmd>
  {\ifthenelse{\isundefined{\@email}}{}{\addtomaketitle{\emailsymbol\emaillink{\@email}}}}% <search>
  {\ifthenelse{\isundefined{\@emailA}}{}{\addtomaketitle{\emailsymbol\emaillink{\@emailA}}}%
   \ifthenelse{\isundefined{\@emailB}}{}{\addtomaketitle{\emailsymbol\emaillink{\@emailB}}}}% <replace>
  {}{}% <success><failure>
\makeatother
 
%----------------------------------------------------------------------------------------
%   NAME AND CONTACT INFORMATION SECTION
%----------------------------------------------------------------------------------------

\firstname{Stephen} % Your first name
\familyname{Taylor} % Your last name

% All information in this block is optional, comment out any lines you don't need
\title{\huge{Teaching Statement}}
\address{Jet Propulsion Laboratory, 4800 Oak Grove Drive}{Pasadena, CA 91109}
\phone[mobile]{+1 (626) 689-5832}
\emailA{Stephen.R.Taylor@jpl.nasa.gov}
%\emailB{steve.taylor1987@gmail.com}
\homepage{stevertaylor.github.io}
\social[github]{stevertaylor}
\social[linkedin][linkedin.com/in/stephen-taylor-a8164787]{stephen-taylor}

%----------------------------------------------------------------------------------------

\begin{document}

\makecvtitle % Print the CV title

We remember our best and worst teachers, such is the indelible impact that teaching can have on us. A good teacher can make a struggling student achieve beyond their imagining, and can set a talented student on the road to greatness. A poor teacher can drag even the brightest student down. Instructing and supervising students is thus an incredible privilege that comes with many responsibilities that I take extremely seriously. These students will form part of an unbroken chain of knowledge transfer --- it is my responsibility to pass on the discoveries of prior generations, whilst also teaching the students to think for themselves so that they can do the same for the next generation.
\vspace{2mm}

In some quarters, teaching is seen as lesser than research, since it is the latter that advances inquiry into the unknown. However, arguably the greatest scientist of the $20^\mathrm{th}$ century once said ``If you can't explain it simply, you don't understand it well enough". Einstein saw that teaching organizes the mind and improves the practice of research by compelling you to break a subject down into easily-digestible logical steps. I have always found this to be true, whether it was helping out my fellow Oxford undergraduate physics students with a difficult new topic, teaching my Cambridge undergraduate students the beauty of special and general relativity, or instructing Caltech graduate students on the latest advances in gravitational-wave data-analysis strategies.

Focus on specific PLANS for courses. Ie, new courses you?d develop AND basic intro/generalist courses and how you?d handle them, pref. in innovative ways.

wide general good --> teaching strategies that manifest this good --> examples from specific classes --> evidence that the strategies were effective --> conclusion

gravitaitonal wave infernece technique classes, show students the links between different detectors and bands, pulsar-timing has borrowed stratgeies from ligo, but pulsar-timing is far more advanced in stochastic signal inference. pulsar-timing has fed back to ligo in terms of pahse coherent mapping.

taught relativity to cambridge part II undergradtes. small groups allowed deep focused learning, open discussion, and extensions beyond set syllabus. students were responsive. 

one studnet still in contact. caltech graduate student. christopher spalding, involved in exoplanetary research.

studnets don't even know what they know. feynman. let students see the connections and the broad applicability of differtnet techniques. insetad og isladns of rote-learned facts, let studetsn appreciate teh deep bedrock of knowledge that connects differnet disciplines.

\end{document}