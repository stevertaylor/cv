%%%%%%%%%%%%%%%%%%%%%%%%%%%%%%%%%%%%%%%%%
% "ModernCV" CV and Cover Letter
% LaTeX Template
% Version 1.1 (9/12/12)
%
% This template has been downloaded from:
% http://www.LaTeXTemplates.com
%
% Original author:
% Xavier Danaux (xdanaux@gmail.com)
%
% License:
% CC BY-NC-SA 3.0 (http://creativecommons.org/licenses/by-nc-sa/3.0/)
%
% Important note:
% This template requires the moderncv.cls and .sty files to be in the same
% directory as this .tex file. These files provide the resume style and themes
% used for structuring the document.
%
%%%%%%%%%%%%%%%%%%%%%%%%%%%%%%%%%%%%%%%%%

%----------------------------------------------------------------------------------------
%   PACKAGES AND OTHER DOCUMENT CONFIGURATIONS
%----------------------------------------------------------------------------------------

\documentclass[11pt,letterpaper,sans]{moderncv} % Font sizes: 10, 11, or 12; paper sizes: a4paper, letterpaper, a5paper, legalpaper, executivepaper or landscape; font families: sans or roman

\moderncvstyle{banking} % CV theme - options include: 'casual' (default), 'classic', 'oldstyle' and 'banking'
\moderncvcolor{blue} % CV color - options include: 'blue' (default), 'orange', 'green', 'red', 'purple', 'grey' and 'black'
%\usepackage[scale=0.8]{geometry} % Reduce document margins
\usepackage[margin=1in]{geometry}
\usepackage{lastpage}
\usepackage{enumitem}
\setlist[enumerate]{topsep=0pt}
\setlength{\parindent}{2em}
%\setlength{\hintscolumnwidth}{3cm} % Uncomment to change the width of the dates column
%\setlength{\makecvtitlenamewidth}{10cm} % For the 'classic' style, uncomment to adjust the width of the space allocated to your name
\cfoot{Page \thepage\ of \pageref{LastPage}}

\def\prd{{Phys. Rev.} D}
\def\PRL{{Phys.Rev.} Lett}
\def\apjl{{Astrophys. J.} Lett}
\def\apj{{Astrophys. J.}}
\def\CQG{{Class. Quantum Grav.}}
\def\aaps{{A\&AS}}
\def\pasj{{PASJ}}
\def\mnras{{MNRAS}} 
\def\aapr{{A\&ARv}}
\def\aap{{A\&A}}
\def\na{{New Astronomy}}
\def\ptp{{Progress of Theoretical Physics}}
\def\apjs{{ApJS}}
\def\araa{{ARA\&A}}
\def\ssr{{Space Sci. Rev.}} 

\usepackage{etoolbox}% http://ctan.org/pkg/etoolbox
\makeatletter
\newcommand*{\emailA}[1]{\def\@emailA{#1}}
\newcommand*{\emailB}[1]{\def\@emailB{#1}}
\patchcmd{\maketitle}% <cmd>
  {\ifthenelse{\isundefined{\@email}}{}{\addtomaketitle{\emailsymbol\emaillink{\@email}}}}% <search>
  {\ifthenelse{\isundefined{\@emailA}}{}{\addtomaketitle{\emailsymbol\emaillink{\@emailA}}}%
   \ifthenelse{\isundefined{\@emailB}}{}{\addtomaketitle{\emailsymbol\emaillink{\@emailB}}}}% <replace>
  {}{}% <success><failure>
\makeatother
 
%----------------------------------------------------------------------------------------
%   NAME AND CONTACT INFORMATION SECTION
%----------------------------------------------------------------------------------------

\firstname{Stephen} % Your first name
\familyname{Taylor} % Your last name

% All information in this block is optional, comment out any lines you don't need
\title{\huge{Research Statement}}
%\address{Jet Propulsion Laboratory, 4800 Oak Grove Drive}{Pasadena, CA 91109}
%\phone[mobile]{+1 (626) 689-5832}
%\emailA{Stephen.R.Taylor@jpl.nasa.gov}
%\emailB{steve.taylor1987@gmail.com}
%\homepage{stevertaylor.github.io}
%\social[github]{stevertaylor}
%\social[linkedin][linkedin.com/in/stephen-taylor-a8164787]{stephen-taylor}

%----------------------------------------------------------------------------------------

\begin{document}

\makecvtitle % Print the CV title
\setlength{\parskip}{1.3ex plus 0.5ex minus 0.3ex}
\vspace{-8mm}

\noindent My research interests centre around the gravitational-wave (GW) study of compact stellar systems, their progenitor populations, and their astrophysical environments. The LIGO/Virgo collaboration's ground-breaking detection of stellar-mass black hole (BHs) binary systems (with major contributions from the Bozeman group) recently inaugurated the field of observational GW astronomy. But much like electromagnetic waves, the GW spectrum is expansive and ripe for discovery. At frequencies 11 orders of magnitude lower than the LIGO sensitivity band is the pulsar-timing array (PTA) band, where the dominant sources are nanohertz-emitting supermassive ($M > 10^8M_\odot$) black-hole binaries (SMBHBs) that are formed during the mergers of massive galaxies. My work has covered both extremes, but recently I have focused on bringing new inference strategies to pulsar-timing efforts as a means of accelerating the detection of nanohertz GWs, studying the ionized interstellar medium, and performing pulsar noise characterization. In the following I discuss my research accomplishments and plans for the future.

\noindent \textbf{\underline{Previous and current research accomplishments}}

\noindent I began my research career by investigating the prospects for double neutron-star systems detected by ground-based laser interferometers to be used as ``standard sirens''. This exploits the potential to directly infer luminosity distances from their gravitational waveforms, rather than relying on the piecewise-constructed electromagnetic cosmic-distance ladder. Instead of searching for electromagnetic counterparts to obtain a redshift, I constructed a hierarchical Bayesian model that used the information in the redshifted chirp-mass to simultaneously infer the neutron-star mass distribution and the Hubble constant. %My findings indicated that with $100$ systems, and provided the width of the neutron-star mass distribution is less than $0.1 M_\odot$, the Hubble constant could be constrained to within $10\%$ using only GW observations. 
I adapted this model to probe the dark-energy equation of state and the progenitor star-formation rate with the potential third-generation Einstein Telescope.

My interests then diversified to the low-frequency GW regime, where rigorous Bayesian analysis was relatively new to the field. A pulsar-timing detection of GWs will need to show evidence for correlated pulse arrival-times across widely separated pulsars, in agreement with the distinctive Hellings and Downs overlap-reduction function. This is accurate only if the stochastic GW background (GWB) is isotropic, and while this assumption is sufficient for initial detection, detailed characterization of the GWB angular distribution (as a tracer of the host-galaxy distribution) will need more flexible models for the overlap-reduction function. I developed analysis tools to probe the spatial correlations between pulsars as a means of inferring the angular distribution of GWB power, and later authored a high-profile \textit{Physical Review Letter} on behalf of the European Pulsar Timing Array (EPTA) collaboration that provided the first such angular power constraints. This was part of a series of publications by the EPTA, in which I formed part of the core analysis team to study GWB and individual binary constraints with the latest pulsar-timing datasets. I recently extended my research of GWB-power anisotropy to introduce phase-coherent mapping of the nanohertz GW sky (not just stochastic signals), which in turn informed my further development of phase-coherent mapping methods for ground-based laser interferometers.

Beyond stochastic signal characterization, I am interested in the PTA prospects for measuring individual black-hole binary systems. I have developed new rapid search strategies for individual binaries, and have introduced the first pipeline to coherently search for eccentric systems. The latter is particularly important since bringing these massive black-holes to milliparsec separations (i.e. nanohertz orbital frequencies) requires stellar loss-cone scattering and/or circumbinary disk accretion, both of which can increase the eccentricity. These dynamical influences bring my research into direct contact with open issues in galaxy formation and evolution, since overcoming the ``final parsec'' problem of massive black-hole binary evolution needs these environmental coupling mechanisms. My ongoing activities within the North American Nanohertz Observatory for Gravitational-waves (NANOGrav) contributed to the first pulsar-timing constraints on the density of stars in galactic cores, circumbinary accretion rates, and the typical post-dynamical-friction binary eccentricity. I also led the JPL/Caltech pulsar-timing group in authoring an \textit{Astrophysical Journal Letter} to investigate the time-to-detection of nanohertz GWs based on current constraints, and how the aforementioned dynamical influences may impact that. My ongoing development of new data-analysis strategies has occurred in parallel with my construction of an open-source pulsar-timing software package, permitting Bayesian GW searches and noise characterization.

\noindent \textbf{\underline{Future research plans}}

\noindent Pulsar-timing has recently constrained the GW strain amplitude at frequencies of $1/$year to be below $10^{-15}$ with 95$\%$ credibility, necessitating the galaxy evolution community to revise their models of the GWB amplitude. While this is a great success, near-future detection is impeded in two main ways, which influence GW searches at opposite ends of current modeling approaches. Over the next $3$-$5$ years, I plan to tackle these weak links, which are listed below with associated objectives. The coherence of these objectives is illustrated in fig 1 within the broader goal of overhauling the sub-optimal status quo of pulsar-timing analysis to accelerate the detection of nanohertz GWs.

\begin{enumerate} [label=\textbf{\Alph*}]  
\setlength\itemsep{0em}
\item \textbf{Profile folding:} At the initial timing-analysis stage, forthcoming limits to pulsar timing precision will not necessarily be radiometer noise, but rather pulse ``jitter'' arising from the difference between an ensemble averaged pulse profile-template and the observed average over a finite number of pulses. Jitter and other ``profile-domain'' processes are not currently modeled in GWB searches, which deal with observation-averaged pulse arrival-times as the fundamental input. The large collecting areas of new telescopes (e.g. MeerKAT, SKA) are such that jitter will likely set the future pulsar-timing noise floor, impeding these (and current) instruments from near-future detection. 
\\\textbf{Objectives:}
\begin{enumerate}[label=\textbf{\arabic*})] 
\setlength\itemsep{0em}
\item Build a hierarchical Bayesian model that extends pulsar-timing inference backward from the current initial arrival-time data to the raw observed pulse profiles. 
\item Expand the model in item \textbf{1} to include multiple pulsars such that GWB searches can be performed simultaneously with pulse-shape inference and jitter mitigation. 
\end{enumerate}
\item \textbf{Stochastic signal modeling:} GWB searches have heretofore used either a two-parameter power-law spectrum, or included an additional low-frequency turnover to model super-efficient binary hardening through couplings to the galactic environment. Since PTAs are at the stage where stringent signal constraints are being placed, it is crucial to improve these by having physically-detailed and well-motivated spectral models. This is currently inhibited by the intractability of an analytic model that includes all the necessary effects, i.e. stellar loss-cone scattering, disk accretion, radiation reaction.
\\\textbf{Objectives:}
\begin{enumerate}[label=\textbf{\arabic*})] 
\setlength\itemsep{0em}
\setcounter{enumii}{2}
\item Exploit Bayesian model emulation with Gaussian processes to build GW-signal models that are informed by detailed numerical population synthesis calculations. 
\item Develop a web database to store simulated SMBHB populations for reproducibility of the results in \textbf{3}, cross-validation of population simulation techniques, and as a community resource. 
\end{enumerate}
\end{enumerate}

\noindent \textbf{Methodology details}

\noindent This action is divided into the following work packages (WP), with associated tasks (T), and project coherence illustrated in Figure 1. 
WP1 (addressing RO1): An open-source Bayesian profile-domain analysis package
Preliminary analysis has demonstrated the potential for ?generative pulsar timing analysis?12 to combat pulse jitter, wherein the pulse profile shape is inferred in tune with all ephemeris and noise parameters. This can additionally model band-dependent profile changes, epoch-to-epoch profile stochasticity, and secular profile evolution. Dr Taylor will collaborate on a high-profile IPTA project to produce a robust profile-domain analysis package. (external collaborators: Dr L. Lentati, Dr R. Shannon, Dr M. Vallisneri, Dr J. A. Ellis)
T1.1: (Months: 0-2) Constructing the profile-domain likelihood 
Code development for hierarchical profile-domain likelihood, incorporating all processes influencing the pulse shape, arrival-time, evolution, and band-dependence. 
T1.2: (Months: 1-2) Simulations on idealized datasets 
Concurrently with the later stages of Task 1.1, an internal mock data challenge will be performed, contrasting results with traditional TOA-domain analysis. Datasets will be idealized, ranging from single epoch to single radio-channel tests. Parallel-tempering MCMC techniques will sample from a form of the profile-domain likelihood that is marginalized over low-level linear parameters (such that the search dimensionality is manageable, e.g. ~O(10-102)).
T1.3: (Months: 2-4) Operating on real datasets 
The full hierarchical (un-marginalized) profile-domain likelihood will be tested on real multi-channel datasets spanning decades of observations [e.g. Parkes J1909-3744 data]. The dimensionality of the likelihood (>O(103)) will require highly efficient sampling. We will use a recently available Hamiltonian Monte Carlo (HMC) No-U-Turn-Sampler (hereon referred to as NUTS) with custom coordinate transformations to avoid the ?Neal?s funnel? problem of hierarchical likelihoods13. As in Task 1.2, we will compare our results to traditional TOA-domain analysis.
WP2 (addressing RO2): Profile-domain GW searches
The methodology of WP1 will be extended to an array of pulsars for SGWB searches, where isolation of profile-domain processes will increase the analysis sensitivity (see Figure 2). (external collaborators: Dr M. Vallisneri, Dr J. A. Ellis)
	T2.1: (Months: 4-6) TOA-domain upper limits
	TOA-domain SGWB constraints are already possible with the marginalized TOA-domain likelihood, however an upper limit has never been derived with the full hierarchical form. We do so here with the latest IPTA data, again using the NUTS from Task 1.3. We will compare the resulting SGWB constraints to conventional marginalized likelihood results, which should be identical under optimized sampling.
	T2.2: (Months: 6-9) Profile-domain upper limits
	Following from Task 2.1, we will swap out the TOA-domain hierarchical likelihood for the profile-domain hierarchical likelihood of WP1. Since the Bayesian analysis is introduced at a stage closer to the raw data, and properly isolates profile-domain processes from signal and noise processes, these SGWB upper limits should be more sensitive than TOA-domain upper limits. We will once again analyse the latest IPTA data to obtain the best-possible upper limits


	T2.3: (Months: 9-12) Profile-domain SGWB searches and detection significance
	Tasks 2.1 and 2.2 do not model the Hellings and Downs correlations between pulsars, since this is not necessary to obtain GW upper limits. The current HMC NUTS code shows inefficient (but not incorrect) parameter space exploration when these are modelled. We will explore alternative coordinate transformations for the low-level parameters, in addition to an adaptive mass matrix for the Hamiltonian trajectories. If these still show inefficiencies, we will explore methods such as Riemannian-manifold HMC or elliptical slice sampling. Regardless of inefficiencies, we will perform the first profile-domain GW search, with an associated posterior odds ratio for detection. The technique will be applied to the latest IPTA data for the best sensitivity and detection prospects.
WP3 (addressing RO3): Bayesian model emulation for gravitational-wave searches
Physically-detailed SGWB spectral models will be developed which can be rapidly computed within either the TOA-domain or profile-domain likelihood. These models will include all parameters that influence the dynamical evolution of SMBHBs through the final parsec of orbital separation, including the degree of stellar loss-cone refilling, binary eccentricity, and binary accretion rate from a circumbinary disk14. (external collaborators: Dr J. R. Gair, Dr L. Sampson)
T3.1: (Months: 12-13) Developing the Bayesian model   
                                       emulation formalism
In Bayesian model emulation, we run a small number of expensive simulations with varying input parameter values (e.g. binary eccentricity at the time of hardening). The output are treated as draws from a Gaussian process (GP), which are used to train the GP kernel hyper-parameters, allowing predictions (with uncertainties) at arbitrary parameter locations15 (see Figure 3 for a demonstration). We will adapt this for SGWB searches, where the model will be a GP trained on synthesised black-hole populations, instead of an analytically-derived signal model.
	T3.2: (Months: 12-15) Detailed SMBHB population synthesis simulations
During Task 3.1, Dr Taylor and Dr Sesana will run detailed population synthesis simulations across the full expanse of the physical parameter space influencing SMBHB evolution in galactic nuclei. Spectra will be constructed from these simulations to train a GP model, which is stored for subsequent inference.
	T3.3: (Months: 15-19) Constraining the final-parsec problem with a GP model
The trained GP model will be used in an analysis of publicly available PTA data (delivering a high-profile short-author publication) and the latest IPTA data (delivering a high-profile collaboration result) to constrain the typical dynamical environment of SMBHBs. 
WP4 (addressing RO4): Web-portal and repository for SMBHB populations
New PTA techniques require simulated binary populations to inject realistic GW signals into pulsar-timing data for validation. There are no standard repositories to find such populations, requiring researchers to obtain them via private communication with a handful of experts. This is an impediment to open science, and the proper cross-validation of the simulation routines used to obtain the populations.
(external collaborators: Prof S. McWilliams, Mr J. Simon, Mr L. Kelley)
	T4.1: (Months: 15-18) Database design and prototype build
	Concurrently with Task 3.3, we will design a SQL database around the binary populations generated by Dr Sesana for WP3. The goals are to ensure that the results of Task 3.3 are reproducible, and more broadly that the PTA community has access to a standard repository of populations for robust signal recovery tests. The prototype database will permit browser queries and direct python queries.
	T4.2: (Months: 18-19) Hosting simulation output from other research groups
Different groups employ different binary simulation techniques. We will host the simulation output from our external collaborators (listed above) to ensure a complete overview of methodologies are represented, and to allow straightforward comparison of simulation products between groups.
	T4.3: (Months: 19-24) Front-end development and paper release
We will work closely with the UoB computer science department to deliver a user-friendly web front-end with a similar layout to the Illustris and EAGLE web-portals. Concurrently with this front-end development, we will draft a paper describing the hosted simulations and the database features. 






\end{document}